\documentclass[a4paper]{article}
\pagestyle{empty}

\usepackage[utf8]{inputenc}
\usepackage[english]{babel}

\usepackage[pdftex]{graphicx}
\graphicspath{{./figures/}}
\DeclareGraphicsExtensions{.pdf,.jpeg,.png}

\usepackage[margin=0mm]{geometry}
\usepackage{tikz}
\usepackage{verbatim}

\begin{document}

% A4 paper size: 210 x 297 mm
% Figure ratio is 10:7 standard, 10:3.5 / 3.5:7 half size

\newcommand{\singlewidthrow}{90mm}
\newcommand{\doublewidthrow}{190mm}
\newcommand{\xleft}{4mm}
\newcommand{\xright}{104mm}
\newcommand{\yone}{-3mm}
\newcommand{\ytwo}{-73mm}
\newcommand{\ythree}{-143mm}
\newcommand{\yfour}{-213mm}

% Draw the following structure:
% -1- -2-
% -3- -4-
% -5- -6-
% -7- -8-

\begin{tikzpicture}[overlay]
	\node[draw, anchor=north west, text width=\singlewidthrow, color=white    ] (fig1) at (\xleft,\yone) {
		\verbatiminput{description.txt}
	};

	\node[draw, anchor=north west, text width=\singlewidthrow, color=lightgray] (fig2) at (\xright,\yone) {
		\includegraphics[width=0.49\textwidth]{sigCovDistHist}
		\includegraphics[width=0.49\textwidth]{rsuSatDistHist}
	};

	\node[draw, anchor=north west, text width=\singlewidthrow, color=lightgray] (fig3) at (\xleft,\ytwo) {
		\includegraphics[width=\textwidth]{covOverTime}
	};

	\node[draw, anchor=north west, text width=\singlewidthrow, color=lightgray] (fig4) at (\xright,\ytwo) {
		\includegraphics[width=\textwidth]{actVehCnt}
		\includegraphics[width=\textwidth]{actRsuCnt}
	};

	% \node[draw, anchor=north west, text width=\singlewidthrow, color=lightgray] (fig5) at (\xleft,\ythree) {
	% };

	\node[draw, anchor=north west, text width=\singlewidthrow, color=lightgray] (fig6) at (\xright,\ythree) {
		\includegraphics[width=\textwidth]{meanSigOvrTime}
		\includegraphics[width=\textwidth]{meanSatOvrTime}
		\includegraphics[width=\textwidth]{sigToSatOvrTime}
	};

	% \node[draw, anchor=north west, text width=\singlewidthrow, color=white    ] (fig7) at (\xleft,\yfour) {
	% };

	% \node[draw, anchor=north west, text width=\singlewidthrow, color=lightgray] (fig8) at (\xright,\yfour) {
	% };
\end{tikzpicture}

\end{document}
